\section{Conclusions}\label{sec:conclusions}
The goal of our research was to create a drone capable of multi-apple detection and
enumeration.
Although we weren't able to create a product capable of accomplishing the initial
goal, we were able to provide a ``proof of concept'' model and intelligence hierarchy
that, with more time and resources, would provide a way to developing a successful
flying drone crop ripeness detector.
The models that we used were consistently lightweight and reportedly capable of
performing with minimal training data.
Although we experienced a decent amount of success using these models, we lacked the
data, training time, and compute capacity to create a model that was consistently
performant and accurate.
We suspect that given a larger, more diverse dataset, we would be able to produce
more accurate models.
We concluded that despite the shortcomings of a drone, such as short battery life and
limited computational power, it was still the best option out of available competing
systems.
This is due to the limited nature of its interactions with its environment.
Wheeled systems, specifically skid steer systems, destroy the terrain that they move
across.
The drone is capable of moving throughout the orchard without affecting its
surroundings in a physical way.

The code for this project can be found in~\cite{FruitFly}.
