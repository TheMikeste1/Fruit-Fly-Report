\section{Introduction}
%  Present topic and goal
Farmers require a lot of land in order to grow their crops, and navigating this land is time consuming and difficult.
In the U.S., apple orchards take up over 382 thousand acres of land, requiring over 100 thousand workers~\cite{USApple}. 
Our project focuses on helping farmers manage crops across their many acres of land by making use of a drone equipped with machine learning models to 
find apples within the orchard and predict their ripeness and health. By avoiding the need for manual inspection of apples over this large area 
we can lower the time demand for farmers, and provide a more standardized method of determining crop health by removing the subjectivity of human decision making.
Although the original scale of the product involved analysis of entire orchards, in order to meet our time constraints, we focused on training a drone to recognize, follow, and determine the ripeness of an apple. 
We succeeded in training a drone to recognize and follow an apple through the use of a Haar-Cascade model in conjunction with an support vector machine (SVM) to determine the apple's ripeness.
This paper functions as a proof of concept and a starting point for future research into detecting fruit in trees and generating predictive yields.
\\
