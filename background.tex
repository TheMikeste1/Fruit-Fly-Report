\section{Background}
\subsection{UAV's in Agriculture}
Although the concept of including Unarmed Aerial Vehicles (UAVs) in agriculture is a rather new concept, there has been some preliminary work done by other researchers on the subject.
Predominately, research conducted in this field has been investigative, rather than relying on actual experimentation.
An example of this is provided in ``The influence of drone monitoring on crop health and harvest size'', a paper outlining possible uses of drones in the agricultural scene \cite{Reinecke2017}.
During the course of the research conducted by the authors of the aforementioned study, they conducted two interviews. 
The first interview was with UVIRCO, a company "who specialize[s] in cameras and often fit them on drones'' and Aerobatics, which is a drone manufacturing company \cite{Reinecke2017}.
The researchers made several conclusions that helped to contribute to our interest in the project.
For examples, they concluded, ``drones can be equipped with a multispectral camera that can detect the water content underground, which can allow a farmer to determine if a crop row is parched or overhydrated'' \cite{Reinecke2017}.
In addition to this, their findings showed that "drones can create a digital map of a field, detect problems with crop health, find missing livestock, find leaks in irrigation systems, detect the size and spread of fires and potentially spray pesticides" \cite{Reinecke2017}.
In another study conducted in 2017, the concept of using drones to monitor soil moisture levels to determine plant health and other information like possible irrigation system leaks \cite{Hassan2017}. In this paper, the authors discuss the alternative approaches of using satellite imaging to solve analogous problems and determined that the use of drones was "a potential step forward for possible future use in precision agriculture and irrigation scheduling" \cite{Hassan2017}.
\\
Because many of the applications of drones in agriculture require special cameras, such as ``multispectral and thermal cameras,'' we sought to make use of standard imaging systems in conjunction with machine learning in order to lower the expense barrier an orchardist would need to cross to get a useful product \cite{Reinecke2017}.