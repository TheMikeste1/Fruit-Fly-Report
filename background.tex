\section{Background}
\subsection{UAV's in Agriculture}
Although the concept of including Unarmed Aerial Vehicles (UAVs) in agriculture is a rather new concept, there has been some preliminary work done by other researchers on the subject.
Predominately, research conducted in this field has been investigative, rather than relying on actual experimentation.
An example of this is provided in "The influence of drone monitoring on crop health and harvest size", a paper outlining possible uses of drones in the agricultural scene \cite{Reinecke2017}.
During the course of the research conducted by the authors of the aforementioned study, they conducted two interviews. 
The first interview was with UVIRCO, a company "who specialize[s] in cameras and often fit them on drones" and Aerobatics, which is a drone manufacturing company \cite{Reinecke2017}.
The researchers made several conclusions that helped to contribute to our interest in the project.
For examples, they concluded, "drones can be equipped with a multi-spectral camera that can detect the water content underground, which can allow a farmer to determine if a crop row is parched or over-hydrated" \cite{Reinecke2017}.
In addition to this, several other research papers were referenced which concluded 
